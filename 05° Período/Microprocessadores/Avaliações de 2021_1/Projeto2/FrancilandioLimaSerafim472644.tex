\documentclass[a4paper, 12pt]{article}
\usepackage[top=2cm, bottom=2cm, left=2.5cm, right=2.5cm]{geometry}
\usepackage[utf8]{inputenc}
\usepackage{amsmath, amsfonts, amssymb}
\usepackage{graphicx}

\begin{document}
\textbf{\underline{ALUNO:} Francilândio Lima Serafim.}

\textbf{\underline{MATRÍCULA:} 472644.}
\begin{enumerate}
	\item Em relação ao Leaky Bucket o tráfego em rajada é tratado de forma que o mesmo é transformado em um fluxo constante, enquanto que esse tráfego é tratado de forma a ter uma taxa máxima regulada por parte do Token Bucket.
	\item O tráfego em rajada é uma variação brusca na taxa de dados que ocorre em um curto intervalo de tempo, que acaba causando congestionamento em uma rede fazendo a conexão tornar-se lenta.
	\item SMTP é um protocolo de transferência de e-mails, mais usado para transferir e-mails de um servidor para outro com conexão ponto a ponto e usa a porta 25. O FTP é o modo mais simples que se usa para transferir arquivos/dados entre dois computadores e usa as portas 20 e 21.
	\item Para decodificar a palavra, fazemos a associação entre a n-ésima letra da chave através da primeira coluna da tabela dada com a n-ésima letra da palavra codificada situadas na mesma linha e associamos à letra da primeira linha que está na mesma coluna da última letra mencionada como sendo a correspondente da palavra original.Como a chave tem uma letra a menos que a palavra codificada, a última letra da palavra é associada à primeira do código, então fazendo o que foi descrito encontramos a palavra SEGUIR, e o procedimento pode ser visto na Figure \ref{fig1}.
	\begin{figure}[!htb]
		\caption{Decodificação por Vigenère.}
		\centering
		\includegraphics[scale=1]{Tabela.png}
		\label{fig1}
	\end{figure}
	\item Prosseguimos pelos passos a seguir:
	\begin{itemize}
		\item $n = p\cdot q = 3\cdot 7 = 21$
		\item $\phi = (p-1)\cdot (q-1) = (3-1)\cdot (7-1) = 12$
	\end{itemize}
	
	Ou seja, temos que $n = 21$, $\phi = 12$, $e = 23$ e $d = 11$, desse modo fazemos:$$original = texto\_cifrado^{d}\mod n$$
	$$original = 11^{11}\mod 21$$
	$$original = 2$$
	Por fim, podemos fazer um teste para verificar o valor encontrado como segue:$$texto\_cifrado = original^e\mod n$$
	$$2^{23}\mod 21 = 11$$
	Logo, o texto original vale 2.
\end{enumerate}
\end{document}